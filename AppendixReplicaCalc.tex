In this chapter we carry out the calculation for the partition function described in section \ref{sec:rle_statmech} and originally presented in \cite{DeMartinoMarsili04} in details. Although the analitical form for the maximization will not be used in the Applications part of this PhD thesis, we believe it's instructive to the reader that is not familiar with the replica trick 

  Nesta seção iremos detalhar o cálculo da função de partição do
  modelo de Economias Lineares via método de réplicas, que foi
  mencionado no texto. Esta seção serve dois propósitos: uma descrição
  detalhada e didática da conta original feita pelos autores em
  \cite{DeMartinoMarsili04}, que acreditamos ser de grande utilidade
  para leitores que desejam acompanhar os detalhes do artigo, e a
  continuação desta conta para $\beta$ arbitrário e no caso de
  desordem \emph{annealed}, ambas extensões feitas como parte da
  pesquisa deste Doutorado.

  A função de partição é dada por

  \begin{equation}
    \label{eq:appZ}
    Z(\beta | \xi, x_0) = \int_0^\infty ds \int_0^\infty dx e^{\beta U(x)} \delta\left(x - x_0 - \sum_i s_i \xi_i \right)
  \end{equation}

  Desejamos calcular o valor esperado de $u = \langle U(x) \rangle$ no
  equilíbrio. Este valor médio é feito sobre três grandezas
  aleatórias, a desordem fixa $\xi, x_0$ e a variável dinâmica $x$.

  \begin{equation}
    \label{eq:20}
    u = \int d\xi dx_0 P(x_0, \xi) \int_0^\infty U(x) p(x) dx
  \end{equation}

  Onde $P(x_0, \xi)$ é a distribuição de probabilidade conjunta da desordem. Escrevemos
  assim por simplicidade, mas lembramos que ela é uma distribuição de
  duas variáveis independentes, e vale que

  \begin{equation}
    \label{eq:24}
    \Prob(x_0 = y) = e^{-y}
  \end{equation}

  e

  \begin{equation}
    \label{eq:25}
    \Prob(\xi_i) = \frac{1}{P_\xi} \prod_{c=1}^M \frac{1}{\sqrt{2\pi \Delta^2}}e^{-\frac{(\xi_i^c)^2}{2\Delta^2}}
    \delta\left(\sum_{c=1}^M \xi_i^c + \epsilon\right),
  \end{equation}

onde $P_\xi$ é a normalização, dada por:

\begin{equation}
  \label{eq:26}
  P_{\xi_i} = \int_0^\infty \prod_{c=1}^M d\xi_i^c \frac{1}{\sqrt{2\pi M^{-1}\Delta}}e^{-\frac{(\xi_i^c)^2}{2 M^{-1}\Delta}}
    \delta\left(\sum_{c=1}^M \xi_i^c + \epsilon\right)
\end{equation}

Para calcular esta integral, devemos usar uma identidade que será
muito útil durante toda a conta. A transformada de Fourier de uma
função $\delta$ de Dirac é dada por

\begin{equation}
  \label{eq:27}
  \delta(x) = \frac{1}{2\pi} \int_{-\infty}^\infty dk e^{i k x}
\end{equation}

Usando essa identidade podemos calcular a normalização $P_\xi$:

\begin{align}
  \label{eq:26}
  P_{\xi_i} & = \int_0^\infty \prod_{c=1}^M d\xi_i^c
  \frac{1}{\sqrt{2\pi M^{-1}\Delta}}e^{-\frac{(\xi_i^c)^2}{2 M^{-1}\Delta}}
  \int_{-\infty}^\infty dk \frac{1}{2\pi} e^{i k (\sum_c \xi_i^c + \epsilon)} =
  \\ & \int_{-\infty}^\infty dk \frac{1}{2\pi} e^{ik\epsilon} \prod_{c=1}^M
  \int_0^\infty d\xi_i^c \frac{1}{\sqrt{2\pi
      M^{-1}\Delta}}e^{-\frac{(\xi_i^c)^2}{2 M^{-1}\Delta} + i k \xi_i^c}
\end{align}

Para resolver a integral gaussiana, usamos outra identidade comum que
vale a pena ser mencionada. Podemos completar os quadrados de qualquer
integral em $e^{-a x^2 + bx}$ da forma:

\begin{equation}
  \label{eq:28}
  \int_{-\infty}^\infty dx e^{-ax^2 + bx} = \sqrt{\frac{\pi}{a}} e^{\frac{b^2}{4a}}
\end{equation}

Esta identidade é muito útil não só para integrar termos como os do
lado esquerdo, mas também para linearizar termos como os do lado
direito.

Usando esta identidade em \eqref{eq:26} temos:

\begin{equation}
  \label{eq:29}
  P_{\xi_i} = \int_{-\infty}^\infty dk \frac{1}{2\pi} e^{ik\epsilon} 
  e^{-M\frac{M^{-1}\Delta k^2}{2}} = \frac{1}{\sqrt{2\pi\Delta}} e^{-\frac{\epsilon^2}{2\Delta}}
\end{equation}


Voltando ao cálculo de $u$, podemos escrever a integral em $dx$ como

  \begin{equation}
    \label{eq:21}
    \int_0^\infty U(x) \frac{e^{\beta U(x)}}{Z} dx = \frac{\del}{\del \beta} \log Z
  \end{equation}

  e temos a nova média sobre a desordem dada por

  \begin{equation}
    \label{eq:20}
    u = \int d\xi dx_0 \Prob(x_0, \xi) \log Z
  \end{equation}


  O nosso problema de calcular $u$ se reduz a calcular $\log
  Z$. Porém, isto não se mostra uma tarefa fácil, especialmente porque
  a média sobre a desordem deve ser feita depois de calcular a função
  de partição. Para tornar o problema tratáveis, utilizamos o método
  de réplicas \cite{NishimoriBook}, que consiste em escrever $\log Z$
  como

  \begin{equation}
    \label{eq:11}
    \log Z = \lim_{r\to 0} \frac{Z^r - 1}{r}
  \end{equation}
  
  E podemos calcular $Z^r$ ao invés de $\log Z$. A técniza utilizada
  para calcular $Z^r$ é tratar $r$ como um número inteiro arbitrário e
  escrever o produto como $Z^r = Z_1 Z_2 \ldots Z_r$, onde cada $Z_a$
  é dito uma \emph{réplica} do sistema, com variáveis dinâmicas $x^a$
  e $s^a$ (que ainda são vetores!) independentes. Assim, escrevemos:

  \begin{equation}
    \label{eq:23}
    Z^r = \int_0^\infty ds^1 \int_0^\infty dx^1 e^{\beta U(x^1)}
    \delta\left(x^1 - x_0 - \sum_i s^1_i \xi_i \right) \ldots \int_0^\infty ds^r \int_0^\infty dx^r e^{\beta U(x^r)}
    \delta\left(x^r - x_0 - \sum_i s^r_i \xi_i \right)
  \end{equation}

Acumulando todos esses termos obtém-se

  \begin{equation}
    \label{eq:22}
    Z^r = \int_0^\infty \prod_{a=1}^r d x_a \int_0^\infty
    \prod_{a=1}^r d s_a e^{\beta \sum_a U(x_a)} \prod_{a=1}^r
    \prod_{c=1}^M \delta \left( x_c^a - x_0^c - \sum_{i=1}^N s_i^a \xi_i^c\right)
  \end{equation}

  Para retomar o estado atual do cálculo de $u$, estamos fazendo:

  \begin{equation}
    \label{eq:20}
    u = \int d\xi dx_0 \Prob(x_0, \xi) \lim_{r\to 0} \frac{Z^r - 1}{r}
  \end{equation}
  
  Aqui usa-se um artifício comum ao método de réplicas que é mudar a
  ordem da integral sobre a desordem e do limite de $r\to 0$. Não há
  demonstração rigorosa que isso pode ser feito, porém é a única forma
  de resolver essas equações. A integral sobre $x_0$ será de pouca
  importância e deixaremos para o final. A integral sobre $\xi$ deve
  ser feita neste ponto por estar multiplicando outro termo de
  integração ($s_i^a$). Fazemos então a integral $\int d\xi \Prob(\xi)
  Z^r$:

  \begin{align}
    \label{eq:30}
    \int d\xi \Prob(\xi)
  Z^r = & \int_{-\infty}^\infty \prod_{c=1}^M \prod_{i=1}^N \frac{1}{P_{\xi_i}} d\xi_i^c \frac{1}{\sqrt{2\pi M^{-1}\Delta}}e^{-\frac{(\xi_i^c)^2}{2 M^{-1}\Delta}}
    \delta\left(\sum_{c=1}^M \xi_i^c + \epsilon\right) \times \\ &
    \times \int_0^\infty \prod_{a=1}^r d x_a \int_0^\infty
    \prod_{a=1}^r d s_a e^{\beta \sum_a U(x_a)} \prod_{a=1}^r
    \prod_{c=1}^M \delta \left( x_c^a - x_0^c - \sum_{i=1}^N s_i^a
      \xi_i^c\right) \nonumber
  \end{align}

Novamente usaremos a transformada de Fourier de $\delta$. Usaremos as
identidades:

\begin{equation}
  \label{eq:31}
  \delta\left(\sum_{c=1}^M \xi_i^c + \epsilon\right) =
  \int_{-\infty}^\infty \frac{1}{2\pi} d\hat{z}_i e^{i \hat{z}_i \left(\sum_{c=1}^M \xi_i^c + \epsilon\right)}
\end{equation}

\begin{equation}
  \label{eq:32}
  \delta \left(x_c^a - x_0^c - \sum_{i=1}^N s_i^a \xi_i^c\right) = \int_{-\infty}^\infty \frac{1}{2\pi} d\hat{x}_c^a e^{i \hat{x}_c^a \left(x_c^a - x_0^c - \sum_{i=1}^N s_i^a \xi_i^c\right)}
\end{equation}

Escrevendo apenas os termos com $\xi_i^c$ na \eqref{eq:30} e fazendo a
integral sobre $d\xi_i^c$ temos, para cada par $i,c$,

\begin{equation}
  \label{eq:34}
  \int_{-\infty}^\infty d\xi_i^c \frac{1}{\sqrt{2\pi
      M^{-1}\Delta}}e^{-\frac{(\xi_i^c)^2}{2 M^{-1}\Delta}} e^{i
    \hat{z}_i \xi_i^c} e^{-\sum_a i \hat{x}_c^a s_i^a \xi_i^c} =
  e^{-\frac{\Delta}{2M} \left(\hat{z}_i - \sum_a \hat{x}_c^a s_i^a\right)^2}
\end{equation}

Onde novamente foi usada a técnica de completar quadrados na integral
gaussiana. Observe que o termo de normalização é cancelado.

Inserindo o produto $\prod_{i,c} e^{-\frac{\Delta}{2M} \left(\hat{z}_i
    - \sum_a \hat{x}_c^a s_i^a\right)^2}$ de volta na equação
\eqref{eq:30} temos:

\begin{align}
  \label{eq:33}
  & \int_{-\infty}^\infty \prod_{i=1}^N 
  \frac{1}{2\pi} d\hat{z}_i \int_{-\infty}^\infty \prod_{a=1}^r
  \frac{1}{2\pi} \prod_{c=1}^M d\hat{x}_c^a \int_0^{\infty} d x^a
  \int_0^\infty ds^a \frac{1}{\left[\frac{1}{\sqrt{2\pi\Delta}}
      e^{-\frac{\epsilon^2}{2\Delta}}\right]^N}  \times \\ 
  & \times \exp\left[\beta \sum_a U(x_a) + i\epsilon\sum_{i=1}^N \hat{z}_i + i\sum_{a=1}^r \sum_{c=1}^M
    \hat{x}_c^a \left(x_c^a - x_0^c\right) -
    \frac{\Delta}{2M}\sum_{i=1}^N \sum_{c=1}^M\left(\hat{z}_i -
      \sum_{a=1}^r \hat{x}_c^a s_i^a \right)^2\right]
\end{align}

Vamos fazer agora uma mudança de variáveis bastante comum neste tipo
de conta, para que se possa desacoplar as integrais nas diferentes
variáveis. Introduzimos:

\begin{equation}
  \label{eq:36}
  \omega_{ab} = \frac{1}{N} \sum_{i=1}^N s_i^a s_i^b \quad \text{e}
  \quad k_a = \frac{1}{N} \sum_{i=1}^N \hat{z}_i s_i^a
\end{equation}

Para fazer as substituições dessas variáveis, usa-se outra técnica
comum, que é multiplicar a equação \eqref{eq:33} por 1, o que não
altera o resultado, mas escrever a unidade como

\begin{equation}
  \label{eq:k}
  1 = \int dk_{a} \delta\left(k_a -
    \frac{1}{N}\sum_{i=1}^N s_i^a\right) = \int dk_a
  d\hat{k}_a \frac{N}{2\pi i} e^{\hat{k}_{a} [N k_a - \sum_i
      s_i^a]}  
\end{equation}

Este é a mesma identidade de escrever a transformada de Fourier como a
sua transformada, com duas pequenas mudanças: primeiro, utiliza-se a
identidade $\delta(x) = \alpha \delta(\alpha x)$ para se escrever
$\delta(k_a - \frac{1}{N} \sum_i s_i^a) = N \delta(N k_a - \sum_i
s_i^a)$. Isso é bastante útil pois os dois termos são de ordem $N$, e
isso será relevante adiante. A segunda mudança é que na transformada
foi feita a mudança de variáveis $\hat{k}_a \to i\hat{k}_a$. No caso
de $\omega_{ab}$, utiliza-se a mesma identidade, com o cuidado de que
é feito $\hat{\omega}_{ab} \to \frac{i}{2} \omega_{ab}$:

\begin{equation}
  \label{eq:37}
  1 = \int d\omega_{ab} \delta\left(\omega_{ab} -
    \sum_{i=1}^N s_i^a s_i^b\right) = \int d\omega_{ab}
  d\hat{\omega}_{ab} \frac{N}{4\pi i} e^{\frac{1}{2}\hat{\omega}_{ab} [N \omega_{ab} - \sum_i
      s_i^a s_i^b]}
\end{equation}

Por simplicidade, serão omitidos agora os limites de integração quando
a integral for $\int_{-\infty}^\infty$. Substituindo as novas
variáveis em \eqref{eq:33}:

\begin{align}
  \label{eq:33}
 Z^r  & =  \int d\omega_{ab}
  d\hat{\omega}_{ab} \frac{N}{4\pi i} e^{N \hat{\omega}_{ab}
    \omega_{ab}} e^{ - \hat{\omega}_{ab}\sum_i
      s_i^a s_i^b} \int dk_{a}
  d\hat{k}_{a} \frac{N}{2\pi i} e^{N \hat{k}_{a}
    k_{a}} e^{ - \hat{k}_{a}\sum_i
      s_i^a} \times \nonumber \\ 
    &\times \int \prod_{i=1}^N 
  \frac{1}{2\pi} d\hat{z}_i \int \prod_{a=1}^r
  \frac{1}{2\pi} \prod_{c=1}^M d\hat{x}_c^a \int_0^{\infty} d x^a
  \int_0^\infty ds^a \frac{1}{\left[\frac{1}{\sqrt{2\pi\Delta}}
      e^{-\frac{\epsilon^2}{2\Delta}}\right]^N}  \times \\ 
  & \times e^{\left[\beta \sum_a U(x_a) + i\epsilon\sum_{i=1}^N \hat{z}_i + i\sum_{a=1}^r \sum_{c=1}^M
    \hat{x}_c^a \left(x_c^a - x_0^c\right) -
    \frac{\Delta}{2M} \sum_{c=1}^M \left(\sum_i \hat{z}_i - 2N\sum_a
      k_a \hat{x}^a_c + N \sum_{a,b} \omega_{ab}\hat{x}_c^a
      \hat{x}_c^b\right)\right]} \nonumber
\end{align}

A vantagem desta troca de variáveis é que as somas sobre $i$
fatorizam-se totalmente e podemos substituir $\sum_i s_i^a$ por $N
s^a$. Podemos colocar um termo $N$ em evidência para todos os termos e
escrever a integral sobre $\omega, \hat{\omega}, k$ e $\hat{k}$ como

\begin{equation}
  \label{eq:38}
  \int Z^r d\xi = \int\prod_{a,b=1}^r N
  \frac{d\omega_{ab} d\hat{\omega}_{ab}}{4\pi i} \int \prod_{a=1}^r N
  \frac{dk_{a} d\hat{k}_{a}}{2\pi i} e^{N h(\omega, \hat{\omega}, k, \hat{k})}
\end{equation}

Para uma certa função $h$. A razão de se fazer isso é porque no limite
termodinâmico de $N\to \infty$ o valor da integral é dominado pelo seu
valor máximo e podemos escrever

\begin{equation}
  \label{eq:39}
  \int d\omega \, d\hat{\omega}\, dk \, d\hat{k} e^{N h(\omega, \hat{\omega},
    k, \hat{k})} = \max_{\omega, \hat{\omega}, k, \hat{k}} e^{N h(\omega, \hat{\omega},
    k, \hat{k})}
\end{equation}

Este método de integração é conhecido como método de ponto de sela e é
uma das estratégias básicas utilizadas ao se fazer este tipo de conta
de réplica devido ao limite termodinâmico. A função $h$ no momento
pode ser dividida em três termos, $h = g_1 + g_2 + g_3$, onde

\begin{equation}
  \label{eq:40}
  g_1 = -\sum_{a,b=1}^r  \frac{1}{2} \hat{\omega}_{ab} \omega_{ab} - \sum_{a=1}^r
  \hat{k}_a k_a 
\end{equation}

\begin{equation}
  \label{eq:41}
  g_2 = \log \int \frac{d\hat{z}}{2\pi} \int_0^\infty \prod_{a=1}^r
  \exp \left[ \frac{1}{2} \sum_{a,b} \hat{\omega}_{ab} s_a s_b +
    \hat{z} \sum_{a=1}^r \hat{k}_a s_a + i\epsilon \hat{z} -
    \frac{\Delta}{2} \hat{z}^2\right] - \log \frac{1}{\sqrt{2\pi\Delta}}
      e^{-\frac{\epsilon^2}{2\Delta}}
\end{equation}

\begin{equation}
  \label{eq:42}
  g_3 = \frac{1}{N} \sum_c \log \int \prod_a \frac{d\hat{x}_a}{2\pi}
  \int_0^\infty \prod_a dx^a e^{\beta \sum_a U(x^a) + i \sum_a
    \hat{x}^a(x^a-x_0^c) - \frac{n\Delta}{2} \sum_{a,b} \hat{x}^a
    \hat{x}^b \omega_{ab} + n\Delta \sum_a \hat{x}^a k_a }
\end{equation}


Para encontrar o valor máximo de $h$, devemos resolver o sistema de
equações

\begin{align}
  \label{eq:43}
  \frac{\del h}{\del \omega_{ab}}& = 0, \quad \quad \frac{\del h}{\del
    \hat{\omega}_{ab}} = 0 \nonumber \\     \frac{\del h}{\del
    k_{a}}& = 0, \quad \quad  \frac{\del h}{\del \hat{k}_a}= 0 
\end{align}

Estas equações são as chamadas equações de ponto de sela do cálculo de
réplicas. Embora o objetivo deste cálculo envolvido seja calcular o
valor médio de $U(x)$ no equilíbrio, as equações de ponto de sela nos
dão informações importantes sobre a relação de várias quantidades do
sistema de interesse.

Vamos fazer aqui uma aproximação bastante comum nestes cálculos. Os
termos $\omega_{ab}$ se referem a um overlap entre duas réplicas
diferentes, ou seja, dois sistemas similares (um é a réplica do outro)
com variáveis dinâmicas que evoluem de forma independente. Sabemos,
através dos argumentos de Teoria de Equilíbrio Geral, que o equilíbrio
é bem definido e único. Portanto, qualquer par de réplicas $a,b$ terão
o mesmo equilíbrio e podemos escrever $\omega_{ab}$ como apenas dois
valores: $\Omega$, caso $a = b$, isto é, $\Omega$ é o segundo momento
$\langle s^2 \rangle$ da variável $s$ de um sistema, e $\omega$ caso
$a\neq b$, isto é, todas as réplicas possuem o mesmo overlap. Esta é a
chamada aproximação de réplicas simétricas, que é uma aproximação
exata neste caso pois sabemos que a economia possui um único
equilíbrio. Neste trabalho de Doutorado serão exploradas diversas
alterações que podem fazer com que a Economia não possua mais um
equilíbrio único, isso deve ser levado em conta ao fazer a aproximação
de réplicas simétricas. 

Escrevendo os parâmetros de ordem de forma explicita utilizando o
$\delta$ de Kronecker, serão supostas as seguintes identidades:

\begin{align}
  \label{eq:44}
  \omega_{ab} & = \Omega \delta_{ab} + \omega (1 - \delta_{ab})
  \nonumber \\
  \hat{\omega}_{ab} & = \hat{\Omega} \delta_{ab} + \hat{\omega}
  (1-\delta_{ab}) \nonumber \\
  k_a & = k \\
  \hat{k}_a & = \hat{k} \nonumber
\end{align}

Substituindo estas identidades nas equações \eqref{eq:40} -
\eqref{eq:42} e tomando o limite de $r\to 0$ temos:

\begin{equation}
  \label{eq:46}
  \lim_{r\to 0} \frac{1}{r} g_1 = -\frac{1}{2}\left(\Omega\hat{\Omega}
    - \omega \hat{\omega}\right) - k\hat{k}
\end{equation}

\begin{equation}
  \label{eq:47}
  \lim_{r\to 0} \frac{1}{r} g_2 = \int dt \frac{1}{\sqrt{2\pi}}
  e^{-\frac{t^2}{2}} \log \int_0^\infty ds \, e^{\frac{\hat{\Omega} -
      \hat{\omega}}{2} s^2 + \left[t\left(\frac{\hat{k}^2}{\Delta} +
      \hat{\omega}\right)^{\frac{1}{2}} +
    i\hat{k}\frac{\epsilon}{\Delta}\right] s}
\end{equation}

\begin{equation}
  \label{eq:47}
  \lim_{r\to 0} \frac{1}{r} g_3 = \int dt \frac{1}{\sqrt{2\pi}}
  e^{-\frac{t^2}{2}} \log \int_0^\infty dx \, e^{\beta U(x) -
    \frac{\left(x-x_0 + \sqrt{n\Delta \omega} t - i n \Delta
        k\right)^2}{2n\Delta(\Omega - \omega)} - \frac{1}{2} \log[2\pi
    n \Delta(\Omega - \omega)]}
\end{equation}

Nas equações acima, $t$ é uma variável aleatória gaussiana de média
zero e variância unitária. Ela aparece através do uso da identidade
\eqref{eq:28} nos termos com $\left(\sum_a s^a\right)^2$.

Possuímos agora um vetor de parâmetros de ordem $\Theta = (\Omega,
\hat{\Omega}, \omega, \hat{\omega}, k, \hat{k})$ e desejamos maximizar
$h$ em função deste vetor, com alguns desejos sobre a solução. Em
particular, ela deve ser bem definida para $\beta \to
\infty$. Novamente pelo equilíbrio ser único, no limite de temperatura
zero todas as réplicas devem atingir o mesmo limite e a distância
entre as réplicas deve ser zero. Devemos ter portanto que

\begin{equation}
  \lim_{\beta\to \infty} \Omega - \omega = \frac{1}{2N} \sum_{i=1}^N \left(s_i^a - s_i^b \right)^2 = 0
\end{equation}

Mas isso implica em certas divergências em $g_2$. Da mesma forma, caso
$\beta \to infty$, desejamos fazer integrações de ponto de sela da
forma $\int_0^\infty dx e^{\beta V}$, e isso implicaria em
divergências de certos termos em $V$. Precisamos fazer mudanças de
escala para que os parâmetros de ordem permaneçam finitos. Por isso
usamos as seguintes mudanças de escala

\begin{align}
  \label{eq:anzats}
  \chi = n \Delta \beta (\Omega - \omega), \quad \hat{\chi} = -\frac{\hat{\Omega} - \hat{\omega}}{\beta}, \quad \kappa = -in\Delta k, \\
  \hat{\kappa} = \frac{i\hat{k}}{\Delta \beta}, \quad \hat{\gamma} = \frac{\hat{\omega}}{\beta^2}
\end{align}

A função $h$ se torna

\begin{align}
  \label{eq:h}
  h &= \frac{1}{2} \left(\Omega \hat{\chi} - \frac{\hat{\gamma}
      \chi}{n \Delta} \right) - \frac{1}{n} \kappa \hat{\kappa} +
  \frac{1}{\beta} \left\langle \log \int_0^\infty ds \,
  e^{\beta \left[-\frac{\hat{\chi}}{2}s^2 + (t \sqrt{\hat{\gamma} - \Delta
      \hat{\kappa}^2} + \hat{\kappa}\epsilon) s\right]}
\right\rangle_t + \nonumber \\ &+
\frac{1}{n\beta}\left\langle \log \int_0^\infty dx \, e^{\beta \left[ U(x) - \frac{(x - x_0 + \kappa +
      \sqrt{n\Delta\Omega}t)^2}{2\chi}\right]} \right\rangle_{t,x_0}
\end{align}

Neste ponto o tratamento a temperaturas finitas diverge do original,
em temperatura zero. No limite $\beta \to \infty$, estas integrais
podem novamente ser resolvidas via integração de ponto de sela e a
expressão final para $h$ é dada por

\begin{align}
  \label{eq:48}
  h(\beta \to \infty)& = \left\langle \max_s \left[-\frac{\hat{\chi}}{2}s^2 + (t \sqrt{\hat{\gamma} - \Delta
      \hat{\kappa}^2} + \hat{\kappa}\epsilon) s\right] \right\rangle_t +
  \frac{1}{2} \left(\Omega \hat{\chi} - \frac{\hat{\gamma}
      \chi}{n\Delta}\right) - \frac{1}{n} \kappa \hat{\kappa} + \nonumber\\ &+
  \frac{1}{n} \left\langle \max_x \left[U(x) - \frac{(x - x_0 + \kappa +
      \sqrt{n\Delta\Omega}t)^2}{2\chi}\right] \right\rangle_{t,x_o}
\end{align}

Substituindo na equação \eqref{eq:48} $x$ e $s$ pelas soluções $x^*$ e
$s^*$ que maximização as expressões correspondentes, temos as
seguintes equações de ponto de sela:

\begin{align}
  \frac{\del h}{\del \Omega} & = \frac{\hat{\chi}}{2} -
  \frac{1}{2\chi} \sqrt{\frac{\Delta}{n\Omega}}\left\langle (x^* - x_0
  + \kappa + t \sqrt{n\Delta\Omega})t\right \rangle_{t,x_0} = 0   \label{eq:omega}\\
  \frac{\del h}{\del \kappa} & = -\frac{1}{n}\hat{\kappa} -
  \frac{1}{n\chi}\left\langle x^* - x_0 + \kappa +
    t\sqrt{n\Delta\Omega} \right \rangle_{t,x_0} = 0  \label{eq:kappa} \\
  \frac{\del h}{\del \hat{\kappa}} & = -\frac{\Delta\hat{\kappa}}{\sqrt{\hat{\gamma} - \Delta
      \hat{\kappa}^2}} \left\langle t s^*\right\rangle_t + \epsilon
  \left\langle s^*\right\rangle_t - \frac{\kappa}{n} = 0 \label{eq:hatkappa}\\
  \frac{\del h}{\del \hat{\gamma}} & = \frac{1}{2\sqrt{\hat{\gamma} - \Delta
      \hat{\kappa}^2}} \left\langle t s^*\right\rangle_t -
  \frac{\chi}{2n\Delta} = 0\label{eq:gamma}\\
  \frac{\del h}{\del \chi} & = - \frac{\hat{\gamma}}{2n\Delta} + \frac{\left\langle (x^* - x_0 + \kappa +
    t\sqrt{n\Delta\Omega})^2 \right \rangle_{t,x_0}}{2n\chi^2} = 0\label{eq:chi}\\
  \frac{\del h}{\del \hat{\chi}} & = -\frac{1}{2} \left\langle (s^*)^2
  \right\rangle_t + \frac{1}{2} \Omega = 0\label{eq:hatchi}
\end{align}

Algumas simplificações podem ser feitas nestas equações. Devido ao
limite $\beta \to \infty$ a equação para $h$ em \eqref{eq:48} possui
dois problemas de maximização. Podemos identificar $x^*$ fazendo
$\frac{\del}{\del x} \left[U(x) - \frac{(x - x_0 + \kappa +
    \sqrt{n\Delta\Omega}t)^2}{2\chi}\right] = 0$, ficando com a
seguinte equação implicita:

\begin{equation}
  \label{eq:55}
  x^* = x : U'(x^*) = \frac{(x - x_0 + \kappa +
      \sqrt{n\Delta\Omega}t)}{\chi}
\end{equation}

Podemos inserir isto nas equações de ponto de sela \eqref{eq:omega} a
\eqref{eq:hatchi} para obter algumas identidades úteis. A equação
\eqref{eq:kappa} fica, em $x = x^*$:

\begin{equation}
  \label{eq:56}
  \hat{\kappa} = -\left \langle U'(x^*) \right \rangle_{t,x_0}
\end{equation}

O que nos permite identificar $\hat{\kappa} = -p$ devido a identidade
para os preços obtidos no problema de otimização do consumidor, na
equação \eqref{eq:6}. A \eqref{eq:hatchi} permite que imediatamente se
identifique 

\begin{equation}
  \label{eq:57}
\Omega = \left\langle (s^*)^2 \right\rangle_t
\end{equation}

Os parâmetros restantes devem ser encontrados através de álgebra
básica. Com $\Omega$ e $p$ definidos, imediatamente sai que


\begin{equation}
  \label{eq:58}
\hat{\chi} = \sqrt{\frac{\Delta}{n\Omega}} \left \langle U'(x^*) t
\right \rangle_{t,x_0}
\end{equation}

 A \eqref{eq:chi} nos dá que 

 \begin{equation}
   \label{eq:59}
\hat{\gamma} =
\Delta \left \langle U'(x^*)^2 \right \rangle_{t,x_0}
\end{equation}

Com isso, podemos escrever a variância de $U'(x)$ como 

\begin{equation}
  \label{eq:60}
\sigma = \sqrt{\hat{\gamma} - \Delta \hat{\kappa}^2} = \sqrt{\Delta
    \left(\left \langle U'(x^*)^2 \right \rangle_{t,x_0} - \left
        \langle U'(x^*) \right \rangle_{t,x_0}^2 \right)}
\end{equation}


Finalmente, temos pela
  \eqref{eq:gamma} que

  \begin{equation}
    \label{eq:61}
    \chi = \frac{n\Delta}{\sigma} \left \langle t s^* \right \rangle_t
  \end{equation}

e pela \eqref{eq:hatkappa}:

\begin{equation}
  \label{eq:62}
  \kappa = p \chi + n\epsilon \left \langle s^* \right \rangle_t
\end{equation}
