The main objective of this thesis was illustrating the similarity of objectives in Statistical Mechanics and Economics, and how the interdisciplinary field arising from applying methods of the first into the latter is promising and worth delving into. In the earlier part of this thesis, we made the case that study of Economics is built on models characterized by the solution of maximization problems, for which Statistical Mechanics offer a wealth of tools suitable for exploration, even in the case of complex systems with a large number of interacting agents. Most importantly, we showed that this is due not to naively applying Physics laws to economic system, but because the foundation of Statistical Physics are insensitive to subject matter. This allows us to drawn upon the body of knowledge built over the last decades that deals with the rich phenomena of phase transitions, spin glasses, critically, universality, etc, to study complex systems of all sorts, including economic ones.

We then presented three specific problems in which we show how this interdisciplinary application can be fruitful. In the problem of the inefficient consumer, we have shown how Statistical Mechanics offer a principled framework for dealing with "irrational" behavior, by treating it as a problem with non zero temperature. In our specific application, we have shown that irrational consumers may increase GDP without an increase in the average utility. We are confident that this same procedure may be applied for most equilibrium situations in economics, and the question this work leaves is: what sort of unexpected behavior can we observe in traditional economic models involving agents with varying rationality?

In the problem of the Input-Output matrices, we have shown how the proper employment of Information Theory let us identify when aggregation of real world data destroyed the information contained in it, allowing for wrong conclusions to be drawn from it. The cautionary tale of the Input-Output matrices serve as a lesson that aggregation methods matter, a lesson applicable to all economic statistics. If one is not aware of this danger, the analysis of economic data may be biased. On the other hand, aware of this pitfall of aggregation, we may develop better methods that preserve crucial data structure.

Finally, in the problem of random trading with inequality, we have shown that one can observe relevant features in zero intelligence based models, offering a perspective of what are the "entropic" stationary states of real world trade dynamics. We have shown that spontaneous trades tend to concentrate the wealth, which in turns leads to reduce the liquidity of a market. This is relevant because it sheds light into the importance of offsetting this "entropic" wealth concentration. Being a minimal model, there is a large amount of possibilities for further work in this direction by studying the effects of additional ingredients from real economies, such as return on available capital, but also one can use the model to test strategies that offset the increase of the inequality, as for example government taxation on purchases, wealth transfer programs, etc. This sort of analysis would deepen the discussion on social policies and provide a testing ground for planners.

Admittedly, the problems approached in this thesis merely scratch the surface of what can be done in the crossroads. It's our belief that Statistical Physics and Economics have an very large potential, capable of shifting the status quo of economics just as other fields, such as psychology, with the works of Tversky and Kahneman, and mathematics, with the work in Game Theory of von Neumann and Nash, have done in the past. This thesis would serve its purpose if it inspires the reader that this is a path worth pursuing.