In this chapter we will discuss the Random Linear Economy model
\cite{DeMartinoMarsili04} presented by Andrea De Martino, Matteo
Marsili and Isaac Pérez Castillo which will be the basis of some of
the applications discussed in the rest of this thesis.

An economy in this model is defined by $M$
goods, $N$ firms, with a technological density parameter of $n = N/M$,
and one representative consumer. The consumer has an initial random
wealth $x_0 = (x_0^1, \ldots, x_0^M)$, $x_0^\mu \geq 0$ drawn
independently from an exponential distribution, and wishes to improve
its welfare in the market according to a separable utility function
$U(x) = \sum_{\mu=1}^M u(x^\mu)$.

Each firm in the market has a random technology
$\xi_i = (\xi_i^1, \ldots, \xi_i^\mu)$, where $\xi_i^\mu<0$ represents
an input and $\xi_i^\mu>0$ represents an output.  Each firm $i$ only
has one technology, and its only decision is the scale $s_i$ at which
it operates this technology, ie, each company represents a
transformation in the space of goods given by the $M$ dimensional
vector $s_i \xi_i$. The elements $\xi_i^\mu$ are drawn from a
$\mathcal{N}(0, M^{-1})$ normal distribution, normalized so that
$\sum_{\mu=1}^M \xi_i^\mu = -\epsilon$, that is, all technologies are
a little inefficient. This is too guarantee that there's no way to
combine two (or more) technologies and produce infinite goods.

This economy is closed and therefore we must have a market clearing
condition: the $N$ dimensional production scale vector $s$ has to be
such that

\begin{equation}
x = x_0 + \sum_{i=1}^N s_i \xi_i
\label{eq:market_clearing}
\end{equation}
ie, all the inputs the firms use have to come from the consumer's
initial endowment. As with traditional General Equilibrium scenarios,
the market clearing condition makes it so that solving the consumer
maximum utility problem, $x^* = \argmax_x U(x)$, simultaneously solves
the firms maximum profits problem,
$s_i^* = \argmax_{s_i} s_i p_i \cdot \xi_i$, with the prices being set
also by the first order condition of the consumer's maximization
problem, $p_\mu = \frac{\partial U}{\partial x_\mu}$. In brief, market
clearing makes so that the firms production and the market prices are
set to satisfy consumer's desired demand, and no actor in the market
has an incentive to deviate from this equilibrium.

Market clearing also carries a strong restriction. If we multiply both
sides of the equation \eqref{eq:market_clearing} by $p$, we get

\begin{equation}
  \label{eq:market_clearing_p}
  p\cdot (x - x_0) = \sum_i s_i p \cdot \xi_i,
\end{equation}

The left side of the above equation has to always be smaller or equal
to zero, because of the budget condition. But the right hand side has
to be always greater or equal to zero, because this term represents
the sum of the individual firms' profits and they can always choose
$s_i = 0$ if a firm is losing money. Therefore, we must have that both
sides are equal to zero, and the consequence is that the agent
completely spends all his available budget (ie,
$p\cdot x = p \cdot x_0$, he has not ``leftover'' cash after choosing
$x$) and that the firms either have zero profit ($p\cdot \xi_i = 0$)
or leave the market ($s_i = 0$).

One of the important implications of equation
\eqref{eq:market_clearing_p} is that we may not have more than $M$
firms active at any given equilibrium realization. This is because
$p\cdot \xi_i = 0$ is an equation on $p_\mu$ with $M$ variables, and
if we have more than $M$ equations, this system has no solution.

The model has some very interesting properties which are described at
length in \cite{DeMartinoMarsili04}. In particular, it's possible to
analytically calulate the distribution probabilities of $x$ and $s$
(and therefore of $p$) and see that all macroscopic quantities derived
from these two quantities depend on the number of firms per good
$n = N/M$. The model displays a regime change at $n=2$, ie, two random
technologies per good. When $n<2$, the market is competitive and the
fraction of active firms $\phi = \sum_i \mathbb{I}(s_i > 0) / N$ is
around $\phi = 0.5$. Because each firm has on average half the goods
as inputs and half as outputs, when $n<2$ you don't have enough firms
to span the whole $M$ dimensional space in order to be able to fine
tune the quantities desired for all the goods.

When $n>2$, however, there are many firms to choose from and
statistically it's possible to choose $M$ linear independent firms
that span the whole good space. In this regime, the market becomes
monopolistic and $\phi$ assymptotically goes to zero with $n$, due to
a saturation of active firms on $M$. 

The change in the model economy's GDP also reflects the qualitative
change in allocation. The authors define the gross product for the
model as the total value of goods produced, that is, the sum of
$(x_\mu - x_0^\mu)p_\mu$ for all goods $\mu$ that are produced, ie,
$x_\mu > x_0^\mu$. However, the market clearing condition
\eqref{eq:market_clearing_p} makes the value of goods produced equal
to the value of goods used as input, so we calculate the GDP $Y$ by
averaging over the absolute value of all trades:

\begin{equation}
  \label{eq:5}
  Y = \frac{\sum_{\mu = 1}^M |x_\mu - x_0^\mu|p_\mu}{2 \sum_{\mu = 0}^M p_\mu},
\end{equation}
where the denominator also includes a normalization for the prices.

What is shown in \cite{DeMartinoMarsili04} is that in the competitive
regime when $n<2$, a new firm will have a significant positive effect
on $Y$, while in the monopolistic regime $n>2$ a new firm will have
negligible impact on the gross product. We will revisit this result
later in this paper.

The Random Linear Economies model is particularly suitable for further
analysis because it's a General Equilibrium setting with few
ingredients, but the introduction of stochastic elements offers a
nontrivial phase transition which is not observed in similar
``simple'' economic models in the literature.