In this chapter we will present in detail and discuss the Random
Linear Economy model \cite{DeMartinoMarsili04} developed by Andrea De
Martino, Matteo Marsili and Isaac Pérez Castillo which will be the
basis for some of the applications discussed in the second part of
this thesis.

There are some reasons why we chose to work with this model in
particular: first, it presents a General Equilibrium Model which has
few ingredients but already displays a rich behavior, including phase
transitions which depend on the number of firms in the
market. Secondly, it is analytically solvable using statistical
mechanics techniques, such as using the replica trick to calculate the
partition function. Therefore, it was ideal for trying new venues of
exploration without the difficulty imposed in trying to prove general
phenomena. 

\section{The Model Ingredients}

An economy in the model is, like the General Equilibrium setting,
composed by two distinct actors: consumers and firms. We assume $N$
firms and one single representative consumer with utility function
$U(x)$ and initial endowment $x_0$. This is a common approximation
when doing equilibria calculation in Economics due to the simplicity:
if we have $J$ consumers with independent utility functions $U_j$ (ie,
$U_j$ never depends on $x_k$, $k\neq j$) and initial endowments
$\omega_j$, then either we do not allow wealth transfers of $\omega_j$
and the optimization problem becomes very complicated, or we allow the
central authority to carry out wealth transfers prior to allocation,
and then the the demands generated by the consumers in this scenario
is equivalent to that of a single representative consumer with utility
function $U_R = \sum_{j=1}^J U_j$ and wealth
$\omega_R = \sum_{j=1}^J \omega_j$.

The representative consumer assumption receives considerable criticism
\cite{Kirman92}, chiefly because disregarding interaction among agents
(via the utility of one depending on the decisions of the others)
washes out the possibility of interactions and the wide range of
important and interesting phenomena that in the statistical physics
community we know to be generated precisely by these interactions
\cite{Bouchaud13}, whereas the representative agent is a mean field
approximation for consumers.

That said, the representative consumer is used in this model precisely
because it generates an energy function which is convex and has a
well defined, unique minimum and the resulting partition function can
be calculated analytically in the zero temperature limit, while at the
same time generating interesting behavior.

The consumer and the $N$ firms will trade $M$ goods, with a
technological density parameter given by $n = N/M$. We assume as
before that the consumer has an initial wealth
$x_0 = (x_0^1, \ldots, x_0^M)$, $x_0^\mu \geq 0$, and wishes to
improve its welfare in the market by using his endowment $x_0$ to
purchase a consumption bundle $x$ according to a separable utility
function $U(x) = \sum_{\mu=1}^M u(x^\mu)$. His initial endowment,
however, is assumed to be random, each $x_0^\mu$ drawn independently
from a exponential distribution with unitary scale, ie,

\begin{equation}
  \label{eq:1}
  P(x_0^\mu) = e^{-x_0^\mu}
\end{equation}

As before, the aim of the consumer in this economy is to solve the
maximization problem

\begin{equation}
  \label{eq:3}
  x^\ast = \argmax_{x} U(x) \text{ s. t. } p\cdot x \leq p\cdot x_0
\end{equation}

In most of the analysis done in this thesis we will treat the
particular case of the consumer's separable utility function as
$u(x_\mu) = \log x_\mu$, although any concave function would work
exhibit similar qualitative behavior. The logarithm is a common choice
for the consumer's utility function because it satisfies some of the
usual properties desired for the consumer behavior in economics:
first, the consumer is \textbf{loss averse}, which means that he will
always prefer a guaranteed amount $a$ of any good to a lottery in
which he can win $a + \delta$ with probability 0.5 and $a - \delta$
with probability 0.5, for any $\delta > 0$. He is loss averse because
the disutility losing $\delta$ is larger than the utility of gaining
$\delta$. In our case, we don't have lotteries, but the principle
holds for two goods: if he has $\bar{x} + \delta$ of good $\mu$ and
$\bar{x} + \delta$ of good $\nu$, he will try to find a company that
trades this excess of good $\nu$ so he can average both goods and in
fact, may even do so at a loss (ie, he ends up with
$\bar{x} - \varepsilon$ for both goods, for some
$\varepsilon < \delta$). Also, with the separable utility as chosen,
there are not complementary or substitute goods, ie, goods for which
the consumer prefers to have more (or less) of one if he has
another. Finally, because $u(0) = -\infty$, the consumer will always
try to obtain a little bit of every good, even if at a great cost,
because nothing is worse than having none of a particular good.

The firms on the other hand have each an $M$-dimensional random
technology $\xi_i = (\xi_i^1, \ldots, \xi_i^M)$, where $\xi_i^\mu<0$
represents an input and $\xi_i^\mu>0$ represents an output. The
production set of each firm is the space of all vectors which are
proportional to $\xi_i$, that is, $\Xi_i = s \xi_i$, $s \geq 0$. This
means that each firm $i$ only has one technology and its only decision
is the scale $s_i$ at which it operates this technology. Once chosen
the scale $s_i$, a company will consume $s_i \xi_i^-$ goods and
produce $s_i \xi_i^+$ goods, where $\xi_i^{\pm}$ are the positive and
negative entries of the $\xi_i$ vector.

The elements $\xi_i^\mu$ are independently drawn from a normal
distribution with zero mean and $\Delta/M$ variance, where $\Delta >
0$, and are normalized so that the
sum over all the goods for a company is fixed at a negative value and
all technologies are a little inefficient. We must have then:

\begin{equation}
  \label{eq:2}
  P(x_i^\mu) = \mathcal{N}(x_i^\mu | 0, \Delta M^{-1}), \quad \sum_{\mu=1}^M
  \xi_i^\mu = -\epsilon
\end{equation}


We normalize the technologies to be inefficient so that we don't have
a combination of firms producing infinite goods, ie, firm $i$
and $j$ can produce infinite amounts of certain goods by each feeding
its output to be used as the other's input. 

The objective of each company in the market is the same as before:
each firm $i$ tries independently to choose it's production scale
$s_i$ as to maximize it's profits:

\begin{equation}
  \label{eq:6}
  s_i^\ast = \argmax_{s_i > 0} p\cdot (s_i \xi_i)
\end{equation}

Other underlying assumptions of General Equilibrium Theory are valid
here: we assume a complete market, where each agent knows the offer
and demand of all other agents, there is no transaction costs and a
good is uniquely defined. Also, agents are price-takers, which means
that they have no power over the prices and must accept them as given.

We also treat the economy as closed and therefore it must satisfy the market
clearing condition. Because we have just one consumption bundle, then
the $N$ dimensional production scale vector $s$ has to be such that

\begin{equation}
x = x_0 + \sum_{i=1}^N s_i \xi_i
\label{eq:market_clearing}
\end{equation}
ie, all the inputs the firms use have to come from the consumer's
initial endowment. 

Because market clearing hold and agents are price takers, we can also
derive the strong restriction on profits discussed before. If we
multiply both sides of the equation \eqref{eq:market_clearing} by $p$,
we get

\begin{equation}
  \label{eq:market_clearing_p}
  p\cdot (x - x_0) = \sum_i s_i p \cdot \xi_i,
\end{equation}

The left side of the above equation has to always be smaller or equal
to zero, because of the budget condition. But the right hand side has
to be always greater or equal to zero, because this term represents
the sum of the individual firms' profits and if a firm is losing money
they can always choose to set $s_i = 0$ and leave the
market. Therefore, we must have that both sides are equal to zero, and
the consequence is that the agent completely spends all his available
budget (ie, $p\cdot x = p \cdot x_0$, he has not ``leftover'' cash
after choosing $x$) and that the firms either have zero profit
($p\cdot \xi_i = 0$) or leave the market ($s_i = 0$).

One of the important implications of equation
\eqref{eq:market_clearing_p} for the Random Linear Economy model is
that we may not have more than $M$ firms active at any given
realization of equilibrium. If the right hand side of equation
\eqref{eq:market_clearing_p} has to be zero, then for every firm
either $s_i = 0$ or $p\cdot \xi_i = 0$. If $\phi$ is the fraction of
firms active in the market, that is

\begin{equation}
  \label{eq:phi_def}
  \phi = \frac{\sum_{i=1}^N \mathds{I}(s_i > 0)}{N},
\end{equation}
then all of them have $p\cdot \xi_i = 0$. Because the price is the
same for all of them, we have $\phi N$ equations of this type, and $M$
unknowns. For this system to have a non-trivial solution (ie,
$p_\mu > 0$ for all $\mu$), it must be that $\phi N \leq M$, which
implies that

\begin{equation}
  \label{eq:4}
  \phi leq \frac{1}{n}
\end{equation}

Having a single representative consumer (or
many consumers but with wealth transfers) has two important
consequences: first, the price vector is entirely determined by the
consumer's demand. This is a result of the first order condition for
the maximization problem. By taking the derivative of equation
\eqref{eq:3} with the proper Lagrange multiplier we get

\begin{equation}
  \label{eq:8}
  \frac{\del U(x)}{\del x_\mu} - \lambda p_\mu = 0 \Rightarrow p_\mu =
  \frac{1}{\lambda x_\mu}
\end{equation}

Furthermore, the market clearing condition binds the optimization
problem of the consumer and the firms. If we substitute equation
\eqref{eq:market_clearing} in the consumer's utility, we get:

\begin{equation}
  \label{eq:7}
  s^\ast = \argmax_{s : s_i \geq 0} U(x_0 + \sum_{i=1}^N s_i \xi_i)
\end{equation}

We can easily check that the zero profit condition is preserved with
this solution. If $s_i$ is in $s^\ast$, the solution for the
consumer's maximization problem, then either $s_i = 0$ or $s_i >
0$. If $s_i = 0$, the condition is satisfied. Otherwise, if $s_i > 0$,
it means that the constraint $s_i \geq 0$ was not enforced and the
derivative at $s_i$ must be zero. We then have

\begin{equation}
  \label{eq:9}
  0 = \frac{\del U}{\del s_i} = \frac{\del U}{\del x} \frac{\del
    x}{\del s_i} = p\cdot xi_i
\end{equation}

Our problem is now considerably reduced: to find the equilibria in
this model economy all we have to do is solve the maximization problem
in equation \eqref{eq:7}.


\section{The Role of Statistical Mechanics} \label{sec:rle_statmech}

If we were to employ standard convex optimization techniques to solve
to solve \eqref{eq:7}, we would be able to find the solution for a
specific realization of $x_0$ and $\xi$ given a fixed $N$ and $M$. But
if we were to calculate quantities of interest such as consumer
utility, average good consumption, average good price, price deviation
among goods, number of active firms, etc, these would all be random
variables which depend on the realization of endowments and
technologies.

This is, of course, a well known behavior in statistical mechanics. We
solve this by treating the case where the system size is very large,
so that these average quantities converge to a single value. This
isn't always the case, but holds for the so called
\emph{self-averaging} systems. In these systems, these average
quantities for large systems converge to an average over the
realizations for smallers systems.

The general approach to finding the equilibrium properties of a
physical system is is to calculate the partition function for a specific
realization of $\xi$, $x_0$:

\begin{equation}
  \label{eq:10}
  Z(\beta | \xi, x_0) = \int dx e^{\beta U(x| x_0, \xi)},
\end{equation}
where $\beta$ is the inverse value of the temperature. From this, we
can calculate the average value of the utility function by taking the
derivative of $\log Z$:

\begin{equation}
  \label{eq:12}
  \left \langle U \right \rangle (\beta | \xi, x_0) = \int_0^\infty dx
  \frac{e^{\beta U(x|\xi, x_0)}}{Z(\beta |\xi, x_0 } U(x|\xi, x_0) = \frac{\del}{\del \beta} \log Z(\beta | \xi, x_0)
\end{equation}

The maximum value for the utility $U(x)$ is equivalent to the average
value on the ground state\footnote{This is true because $U(x)$ is
  convex and therefore has only one maximum.}, ie:

\begin{equation}
  \label{eq:13}
  \max_x U(x | \xi, x_0) = \lim_{\beta\to\infty} \langle U \rangle (\beta | \xi, x_0)
\end{equation}

However, we are still calculating the maximum as a function of the
samples $x_0$ and $\xi$. In order to get the average behavior, which
holds for a large system, we must average the utility over the
disorder. Assembling all pieces together, we finally get the solution
to equation \eqref{eq:7}:

\begin{equation}
  \label{eq:14}
  \max_x U(x) = \int d\xi dx_0 P(\xi) P(x_0)  \lim_{\beta\to\infty} \frac{\del}{\del \beta} \log \int dx e^{\beta U(x| x_0, \xi)}
\end{equation}
 
The explicit calculation of the expression above is considerably
involved and makes use of a method commonly known as \textbf{replica
  trick} in the statistical physics community. We leave the lengthy calculation for Appendix \ref{sec:appendix_replica} and skip to its solution. The
solution of this calculation is given by

\begin{equation}
  \label{eq:52}
  \lim_{N\to \infty} \frac{1}{N} \max_x \, U(x) = \max_\theta h(\Omega, \kappa,
  p, \sigma, \chi, \hat{\chi}),
\end{equation}
where $\theta = (\Omega, \kappa,
  \sigma, \chi, \hat{\chi})$ are order parameters that arise during
  the calculation and $h$ is given by:

\begin{align}
  \label{eq:53}
  h(\Omega, \kappa, p, \sigma, \chi, \hat{\chi})& = \left\langle
    \max_s \left[-\frac{\hat{\chi}}{2}s^2 + (t \sigma - \epsilon p)
                                                  s\right]
                                                  \right\rangle_t +
                                                  \nonumber\\ & +
  \frac{1}{2} \left(\Omega \hat{\chi} + \frac{\kappa p}{n}\right) -
  \frac{1}{2n\Delta} \chi \sigma^2 - \frac{1}{2n} \chi p^2 + \nonumber\\ &+
  \frac{1}{n} \left\langle \max_x \left[U(x) - \frac{(x - x_0 + \kappa +
      \sqrt{n\Delta\Omega}t)^2}{2\chi}\right] \right\rangle_{t,x_o},
\end{align}
where $t$ is a normal random variable with zero mean and unitary variance.


\section{Results}

The model has some very interesting properties which are described at
length in \cite{DeMartinoMarsili04}. In particular, it's possible to
analytically calulate the distribution probabilities of $x$ and $s$
(and therefore of $p$) and see that all macroscopic quantities derived
from these two quantities depend on the number of firms per good
$n = N/M$. The model displays a regime change at $n=2$, ie, two random
technologies per good. When $n<2$, the market is competitive and the
fraction of active firms $\phi = \sum_i \mathbb{I}(s_i > 0) / N$ is
around $\phi = 0.5$. Because each firm has on average half the goods
as inputs and half as outputs, when $n<2$ you don't have enough firms
to span the whole $M$ dimensional space in order to be able to fine
tune the quantities desired for all the goods.

When $n>2$, however, there are many firms to choose from and
statistically it's possible to choose $M$ linear independent firms
that span the whole good space. In this regime, the market becomes
monopolistic and $\phi$ assymptotically goes to zero with $n$, due to
a saturation of active firms on $M$. 

The change in the model economy's GDP also reflects the qualitative
change in allocation. The authors define the gross product for the
model as the total value of goods produced, that is, the sum of
$(x_\mu - x_0^\mu)p_\mu$ for all goods $\mu$ that are produced, ie,
$x_\mu > x_0^\mu$. However, the market clearing condition
\eqref{eq:market_clearing_p} makes the value of goods produced equal
to the value of goods used as input, so we calculate the GDP $Y$ by
averaging over the absolute value of all trades:

\begin{equation}
  \label{eq:5}
  Y = \frac{\sum_{\mu = 1}^M |x_\mu - x_0^\mu|p_\mu}{2 \sum_{\mu = 0}^M p_\mu},
\end{equation}
where the denominator also includes a normalization for the prices.

What is shown in \cite{DeMartinoMarsili04} is that in the competitive
regime when $n<2$, a new firm will have a significant positive effect
on $Y$, while in the monopolistic regime $n>2$ a new firm will have
negligible impact on the gross product. We will revisit this result
later in this paper.

The Random Linear Economies model is particularly suitable for further
analysis because it's a General Equilibrium setting with few
ingredients, but the introduction of stochastic elements offers a
nontrivial phase transition which is not observed in similar
``simple'' economic models in the literature.