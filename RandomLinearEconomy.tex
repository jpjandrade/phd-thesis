In this chapter we will present in detail and discuss the Random
Linear Economy model \cite{DeMartinoMarsili04} developed by Andrea De
Martino, Matteo Marsili and Isaac Pérez Castillo which will be the
basis for some of the applications discussed in the second part of
this thesis.

There are some reasons why we chose to work with this model in
particular: first, it presents a General Equilibrium Model which has
few ingredients but already displays a rich behavior, including phase
transitions which depend on the number of firms in the
market. Secondly, it is analytically solvable using statistical
mechanics techniques, such as using the replica trick to calculate the
partition function. Therefore, it was ideal for trying new venues of
exploration without the difficulty imposed in trying to prove general
phenomena. 

An economy in the model is, like the General Equilibrium setting,
composed by two distinct actors: consumers and firms. We assume $N$
firms and one single representative consumer, which is a common
approximation when doing equilibria calculation in Economics due to
the simplicity: if we have $J$ consumers with independent
utility functions $U_j$ (ie, $U_j$ never depends on $x_k$, $k\neq j$)
and initial endowments $\omega_j$, then either we do not allow wealth
transfers of $\omega_j$ and the optimization problem becomes very
complicated, or we allow the central authority to carry out wealth
transfers prior to allocation, and then the the demands generated by
the consumers in this scenario is equivalent to that of a single
representative consumer with utility function $U_R = \sum_{j=1}^J U_j$
and wealth $\omega_R = \sum_{j=1}^J \omega_j$.

This assumption receives considerable criticism \cite{Kirman92},
chiefly because disregarding interaction among agents (via the utility
of one depending on the decisions of the others) washes out the
possibility of interactions and the wide range of important and
interesting phenomena that in the statistical physics community we
know to be generated precisely by these interactions
\cite{Bouchaud13}, whereas the representative agent is a mean field
approximation for consumers.

That said, the representative consumer is used in this model precisely
because it generates an energy function which is convex and has a
well defined, unique minimum and the resulting partition function can
be calculated analytically in the zero temperature limit, while at the
same time generating interesting behavior.

The consumer and the $N$ firms will trade $M$ goods, with a
technological density parameter given by $n = N/M$. We assume as
before that the consumer has an initial wealth $x_0 = (x_0^1, \ldots,
x_0^M)$, $x_0^\mu \geq 0$, and wishes to improve
its welfare in the market according to a separable utility function
$U(x) = \sum_{\mu=1}^M u(x^\mu)$. His initial endowment, however, is
assumed to be random, each $x_0^\mu$ drawn independently from a
exponential distribution with unitary scale, ie,

\begin{equation}
  \label{eq:1}
  P(x_0^\mu) = e^{-x_0^\mu}
\end{equation}

The firms on the other hand have each an $M$-dimensional random
technology $\xi_i = (\xi_i^1, \ldots, \xi_i^M)$, where $\xi_i^\mu<0$
represents an input and $\xi_i^\mu>0$ represents an output. The
production set of each firm is the space of all vectors which are
proportional to $\xi_i$, that is, $\Xi_i = s \xi_i$, $s \geq 0$. This means
that each firm $i$ only has one technology and its only decision is
the scale $s_i$ at which it operates this technology. Once chosen the
scale $s_i$, a company will consume $s_i \xi_i^-$ goods and produce
$s_i \xi_i^+$ goods, where $\xi_i^{\pm}$ are the positive and negative
entries of the $\xi_i$ vector.

The elements $\xi_i^\mu$ are independently drawn from a normal
distribution with zero mean and $\Delta/M$ variance, where $\Delta >
0$, and are normalized so that the
sum over all the goods for a company is fixed at a negative value and
all technologies are a little inefficient. We must have then:

\begin{equation}
  \label{eq:2}
  P(x_i^\mu) = \mathcal{N}(x_i^\mu | 0, \Delta M^{-1}), \quad \sum_{\mu=1}^M
  \xi_i^\mu = -\epsilon
\end{equation}


The inefficiency normalization is to avoid having combination of firms
that can produce infinite goods, ie, firm $i$ and $j$ can produce
infinite amounts of certain goods by each feeding its output to be
used as the other's input.

The same underlying principles in General Equilibrium are valid here:
we assume a complete market, where each agent knows the offer and
demand of all other agents, there is no transaction costs and a good
is uniquely defined. Also, agents are price-takers, which mean that
they have no power over the prices and must accept them as given.

This economy is also closed and therefore it must satisfy the market
clearing condition. Because we have just one consumption bundle, then
the $N$ dimensional production scale vector $s$ has to be such that

\begin{equation}
x = x_0 + \sum_{i=1}^N s_i \xi_i
\label{eq:market_clearing}
\end{equation}
ie, all the inputs the firms use have to come from the consumer's
initial endowment. 

Because market clearing hold and agents are price takers, we can also
derive the strong restriction on profits discussed before. If we
multiply both sides of the equation \eqref{eq:market_clearing} by $p$,
we get

\begin{equation}
  \label{eq:market_clearing_p}
  p\cdot (x - x_0) = \sum_i s_i p \cdot \xi_i,
\end{equation}

The left side of the above equation has to always be smaller or equal
to zero, because of the budget condition. But the right hand side has
to be always greater or equal to zero, because this term represents
the sum of the individual firms' profits and if a firm is losing money
they can always choose to set $s_i = 0$ and leave the
market. Therefore, we must have that both sides are equal to zero, and
the consequence is that the agent completely spends all his available
budget (ie, $p\cdot x = p \cdot x_0$, he has not ``leftover'' cash
after choosing $x$) and that the firms either have zero profit
($p\cdot \xi_i = 0$) or leave the market ($s_i = 0$).

One of the important implications of equation
\eqref{eq:market_clearing_p} for the Random Linear Economy model is
that we may not have more than $M$ firms active at any given
equilibrium realization. If the right hand side of equation
\eqref{eq:market_clearing_p} has to be zero, then for every firm
either $s_i = 0$ or $p\cdot \xi_i = 0$. If $\phi$ is the fraction of
firms active in the market, that is

\begin{equation}
  \label{eq:phi_def}
  \phi = \frac{\sum_{i=1}^N \mathds{I}(s_i > 0)}{N},
\end{equation}
then all of them have $p\cdot \xi_i = 0$. Because the price is the
same for all of them, we have $\phi N$ equations of this type, and $M$
unknowns. For this system to have a non-trivial solution (ie,
$p_\mu > 0$ for all $\mu$), it must be that $\phi N \leq M$, which
implies that

\begin{equation}
  \label{eq:4}
  \phi leq \frac{1}{n}
\end{equation}


Because we have a single consumer, the market clearing condition also
completely binds the makes
it so that solving the consumer maximum utility problem,
$x^* = \argmax_x U(x)$, simultaneously solves the firms maximum
profits problem, $s_i^* = \argmax_{s_i} s_i p_i \cdot \xi_i$, with the
prices being set also by the first order condition of the consumer's
maximization problem, $p_\mu = \frac{\partial U}{\partial x_\mu}$. In
brief, market clearing makes so that the firms production and the
market prices are set to satisfy consumer's desired demand, and no
actor in the market has an incentive to deviate from this equilibrium.



The model has some very interesting properties which are described at
length in \cite{DeMartinoMarsili04}. In particular, it's possible to
analytically calulate the distribution probabilities of $x$ and $s$
(and therefore of $p$) and see that all macroscopic quantities derived
from these two quantities depend on the number of firms per good
$n = N/M$. The model displays a regime change at $n=2$, ie, two random
technologies per good. When $n<2$, the market is competitive and the
fraction of active firms $\phi = \sum_i \mathbb{I}(s_i > 0) / N$ is
around $\phi = 0.5$. Because each firm has on average half the goods
as inputs and half as outputs, when $n<2$ you don't have enough firms
to span the whole $M$ dimensional space in order to be able to fine
tune the quantities desired for all the goods.

When $n>2$, however, there are many firms to choose from and
statistically it's possible to choose $M$ linear independent firms
that span the whole good space. In this regime, the market becomes
monopolistic and $\phi$ assymptotically goes to zero with $n$, due to
a saturation of active firms on $M$. 

The change in the model economy's GDP also reflects the qualitative
change in allocation. The authors define the gross product for the
model as the total value of goods produced, that is, the sum of
$(x_\mu - x_0^\mu)p_\mu$ for all goods $\mu$ that are produced, ie,
$x_\mu > x_0^\mu$. However, the market clearing condition
\eqref{eq:market_clearing_p} makes the value of goods produced equal
to the value of goods used as input, so we calculate the GDP $Y$ by
averaging over the absolute value of all trades:

\begin{equation}
  \label{eq:5}
  Y = \frac{\sum_{\mu = 1}^M |x_\mu - x_0^\mu|p_\mu}{2 \sum_{\mu = 0}^M p_\mu},
\end{equation}
where the denominator also includes a normalization for the prices.

What is shown in \cite{DeMartinoMarsili04} is that in the competitive
regime when $n<2$, a new firm will have a significant positive effect
on $Y$, while in the monopolistic regime $n>2$ a new firm will have
negligible impact on the gross product. We will revisit this result
later in this paper.

The Random Linear Economies model is particularly suitable for further
analysis because it's a General Equilibrium setting with few
ingredients, but the introduction of stochastic elements offers a
nontrivial phase transition which is not observed in similar
``simple'' economic models in the literature.