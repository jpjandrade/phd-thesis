%*******************************************************
% Abstract
%*******************************************************
\pdfbookmark[1]{Abstract}{Abstract}
\chapter*{Abstract}

In this thesis, we explore the potential of employing Statistical Mechanics techniques to study economic systems, showing how such an approach could greatly contribute by allowing the study of very complex systems, exhibiting rich behavior such as phase transitions, criticality and glassy phases, which are not found in the usual economic models. We exemplify this potential via three specific problems: \textit{(i)} a Statistical Mechanics framework for dealing with irrational consumers, in which the rationality is set by a parameter akin to a temperature which controls deviations from the maximum of his utility function. We show that an irrational consumer increases the economic activity while decreasing his own utility; \textit{(ii)} an analysis using Information Theory of real world Input-Output matrices, showing that the aggregation methods used to build them most likely underestimated the dependency of the production chain on a few crucial sectors, having important consequences for the analysis of these data; \textit{(iii)} a zero intelligence model in which agents with a power law distributed initial wealth randomly trade goods of different prices. We show that this initial inequality generates a higher inequality in free cash, reducing the overall liquidity in the economy and slowing down the number of trades. We discuss the insights obtained with these three problems, along with their relevance for the larger picture in Economics.