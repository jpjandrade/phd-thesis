%*******************************************************
% Abstract
%*******************************************************
\pdfbookmark[1]{Abstract}{Abstract}
\chapter*{Resumo}

Nesta tese, exploramos o potencial de ser usar técnicas de Mecânica Estatística para o estudo de sistemas econômicos, mostrando como tal abordagem pode contribuir significativamente ao permitir o estudo de sistemas complexos que exibem comportamentos ricos como transições de fase, criticalidade e fases vítreas, não encontradas normalmente em modelos econômicos tradicionais. Exemplificamos este potencial através de três problemas específicos: \textit{(i)} um framework de Mecânica Estatística para lidar com consumidores irracionais, no qual a racionalidade é controlada pela temperatura do sistema, que define o tamanho dos desvios do estado de máxima utilidade. Mostramos que um consumidor irracional aumenta a atividade econômica ao mesmo tempo que diminui seu próprio bem estar; \textit{(ii)} uma anáise usando Teoria da Informação de matrizes Input-Output de economias reais, mostrando que os métodos de agregação utilizados para construí-las provavelmente subestima a dependência das cadeias de produção em certos setores cruciais, com consequências importantes para a analíse destes dados; \textit{(iii)} um modelo em que agentes com uma riqueza inicial distributida como lei de potências trocam aleatoriamente objetos com preços distintos. Mostramos que esta desigualdade inicial gera uma desigualdade ainda maior em dinheiro livre, reduzindo a liquidez total na economia e diminuindo a quantidade de trocas. Discutimos as consequências dos resultados destes três problemas, bem como sua relevância na perspectiva geral em Economia.