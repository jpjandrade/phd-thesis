In this chapter we will discuss some of the economic theories behind
the work of the thesis. Mainly, we will introduce General Equilibrium
Theory and some of its main theorems.

General Equilibrium Theory is a mature and consolidated field of
Economics \cite{Arrow54, mascolell, mckenzie} which aims to
characterize the existence and properties of equilibria in certain
market settings. Economic systems are assumed to frequently have
actors with opposing goals: the owner of a good wants to sell it for
the highest possible price, while its potential buyers would like to
purchase it for as low as possible. Fishermen would like to catch as
many fishes as possible as long as their peers care to not also overdo
it otherwise they may extinguish the oceans. In this way, one expects
economies and markets to converge to a certain steady state and among
other things, General Equilibrium Theory characterizes these steady
states in a rigorous manner. In this sense, it's also a theory in
Microeconomic, because it explains macrobehavior from the incentives
of microscopic agents.

The exposition in this chapter is mostly adapted from \cite{mascolell}
and will be considerably more formal than the rest of this
thesis. This is due to the way the discipline is commonly studied, as
is in itself an interesting fact which we will discuss 

In General Equilibrium Theory, an economy is defined through the
following components: we assume $J$ consumers, $N$ firms and $M$
goods. Each consumer has a consumption set $X_j$ which contains all
possible consumption bundles $x_j = (x_j^1, \ldots x_j^M)$ that the
consumer has access to, ie, each bundle $x_j$ is a $M-$dimensional
vector with nonnegative entries (we are assuming he cannot consume a
negative amount of a good). $X_j$ is limited by ``physical''
contraints, such as no access to water or bread, but not monetary
constraints, which will arise later.

The consumer also has an utility function $U_j(x)$ that takes every
element $x_j \in X_j$ to a real number, representing how much the
consumer values each bundle of his consumption set. This allows us to
define define a preference relationship over the elements in $X_j$
(ie, if the consumer prefers bundle $x$ to $x'$), which is
\emph{complete}\footnote{For every $x, x' \in X_j$, either
  $U_j(x) \geq U_j(x')$ or $U_j(x) \leq U_j(x')$.} and
\emph{transitive}\footnote{For every $x, y, z \in X_j$, if
  $U_j(x) \geq U_j(y)$ and $U_j(y) \geq U_j(z)$, then
  $U_j(x) \geq U_j(z)$.}, two standard requirements in Economics for
rational behavior.

Finally, the consumer is also endowed with an initial bundle of goods
$\omega_j = (\omega_j^1, \ldots, \omega_j^M)$, $\omega_j^\mu \geq 0$
which will define his budget given a set of prices for the goods and
will constraint his choices on $X_j$.

Each firm $i$ has a production set $\Xi_i$ of technologies
$\xi_i = (\xi_i^1, \ldots, \xi_i^M)$ which it is able to operate. Unlike
consumption bundles, which are final allocations and therefore must be
nonnegative, technologies can be any real number: the negative entries
are inputs and the positive entries are outputs that the firm can
operate. $\Xi_i$ is also limited only by ``physical'' constraints, not
by monetary constraints. A firm that has $\xi_i = (-1, 2)$ in its
production set is able to transform one unit of good 1 into two units
of good 2. It won't necessarily be able to transform two units of good
1 into four units of good 2, for that it must also have $\xi'_i = (-2,
4)$ in $\Xi_i$. It might be the case, for example, that companies get more efficient
with production and therefore it might have $\xi''_i = (-2, 6)$ in its
production set.

In General Equilibrium Theory, an economy is formally defined as the tuple

\begin{equation}
  E = \left(\{(X_j,U_j)\}_{j=1}^J,
    \{\Xi_i\}_{i=1}^N, \{\omega_j\}_{j=1}^J \right).\label{eq:econGE}
\end{equation}

One of the theory's assumptions is that the economy described is
\emph{complete}, that is, every agent can exchange every good with no
transaction costs and complete information about the firm's
technologies, other consumer's consumption, etc. Also, a good $\mu$
contains all the possible information that a consumer would take into
account when making his choice. That is, among the space of goods we
could have ``umbrella'' and ``chocolate'', or we could also have ``an
umbrella on August 13th, 2016 in Sao Paulo with 50\% chance of rain''
and ``an umbrella on December 12th, 2016 in Chicago with 90\% chance
of rain''. 

It's assumed that agents are \emph{price-takers}, that is, they are
unable to affect the market prices and therefore take them as a
given. The prices of the goods are given by a $M-$dimensional vector
$p = (p_1, \ldots, p_M)$, where each price is a strictly positive
quantity, ie, $p_\mu>0$ for all $\mu$. This assumes that goods have global
prices, which is consistent with the completeness assumption: there is
no reason why the market prices should be different for certain
consumers or firms if they have complete knowledge and no transaction
costs.

With a price vector $p$ defined, we say the consumer $j$ has a budget
$B_j = p\cdot \omega_j$, which is the monetary value of his initial
endowment. Any bundle he chooses to purchase will cost him
$p\cdot x_j$. His objective, therefore, is to find the best bundle
$x_j$ he is able to afford, that is:

\begin{equation}
  \label{eq:consumer_obj}
  \max_{x_j \in X_j} U(x_j), \quad \text{s.t. } p\cdot x_j \leq p\cdot \omega_j
\end{equation}

The firms, on the other hand, have an operating profit for each
technology given by $p \cdot \xi_i$, which is how much money they earn
by selling their outputs ($\xi_i^\mu > 0$) minus how much they spend
purchasing the inputs ($\xi_i^\mu < 0$). Their objective is to maximize
their profits, that is:

\begin{equation}
  \label{eq:firm_obj}
  \max_{\xi_i \in \Xi_i} p \cdot \xi_i
\end{equation}







% general equilibrium theory
% def of walrasian eq, market clearing, welfare theorems
% criticisms
% diff between economic equilibria and 
\cite{thurner} \cite{Schumpeter39} \cite{savage} \cite{mascolell}
\cite{Roughgarden10, Gintis07}
\cite{Tversky74, Kahneman79, Tversky81}
