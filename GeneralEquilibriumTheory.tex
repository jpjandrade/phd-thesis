In this chapter we will discuss some of the economic theories behind
the work of the thesis. Mainly, we will introduce General Equilibrium
Theory and some of its main results and statements. We will then
discuss how it relates to Statistical Physics and some of its
criticism in the Economics literature.

General Equilibrium Theory is a mature and consolidated field of
Economics \cite{Arrow54, mascolell, mckenzie} which aims to
characterize the existence and properties of equilibria in certain
market settings. Economic systems are assumed to frequently have
actors with opposing goals: the owner of a good wants to sell it for
the highest possible price, while its potential buyers would like to
purchase it for as low as possible. Fishermen would like to catch as
many fishes as possible as long as their peers care to not also overdo
it otherwise they may extinguish the oceans. In this way, one expects
economies and markets to converge to a certain steady state and among
other things, General Equilibrium Theory characterizes these steady
states in a rigorous manner. In this sense, it's also a theory in
Microeconomic, because it explains macrobehavior from the incentives
of microscopic agents.

While the work in this thesis does not aim to strictly expand
economies as they are describe in General Equilibrium Theory, it's
purposes and questions are very similar to the ones physicists usually
go for when studying an economic system: namely, the characterization
of it's equilibrium state. It's important, therefore, to have a
deeper understanding of how it's done in economics, what are the main
concerns and assumptions.

\section{A Brief Exposition}

The exposition in this chapter is mostly adapted and simplified from
\cite{mascolell} and will be considerably more formal than the rest of
this thesis. This is due to the way the discipline is commonly
studied.

In General Equilibrium Theory, an economy is defined through the
following components: we assume $J$ consumers, $N$ firms and $M$
goods. Each consumer has a consumption set $X_j$ which contains all
possible consumption bundles $x_j = (x_j^1, \ldots x_j^M)$ that the
consumer has access to, ie, each bundle $x_j$ is a $M-$dimensional
vector with nonnegative entries (we are assuming he cannot consume a
negative amount of a good). $X_j$ is limited by ``physical''
contraints, such as no access to water or bread, but not monetary
constraints, which will arise later.

The consumer also has an utility function $U_j(x)$ that takes every
element $x_j \in X_j$ to a real number, representing how much the
consumer values each bundle of his consumption set. This allows us to
define define a preference relationship over the elements in $X_j$
(ie, if the consumer prefers bundle $x$ to $x'$), which is
\textbf{complete}\footnote{For every $x, x' \in X_j$, either
  $U_j(x) \geq U_j(x')$ or $U_j(x) \leq U_j(x')$.} and
\textbf{transitive}\footnote{For every $x, y, z \in X_j$, if
  $U_j(x) \geq U_j(y)$ and $U_j(y) \geq U_j(z)$, then
  $U_j(x) \geq U_j(z)$.}, two standard requirements in Economics for
rational behavior.

Finally, the consumer is also endowed with an initial bundle of goods
$\omega_j = (\omega_j^1, \ldots, \omega_j^M)$, $\omega_j^\mu \geq 0$
which will define his budget given a set of prices for the goods and
will constraint his choices on $X_j$.

Each firm $i$ has a production set $\Xi_i$ of technologies
$\xi_i = (\xi_i^1, \ldots, \xi_i^M)$ which it is able to operate. Unlike
consumption bundles, which are final allocations and therefore must be
nonnegative, technologies can be any real number: the negative entries
are inputs and the positive entries are outputs that the firm can
operate. $\Xi_i$ is also limited only by ``physical'' constraints, not
by monetary constraints. A firm that has $\xi_i = (-1, 2)$ in its
production set is able to transform one unit of good 1 into two units
of good 2. It won't necessarily be able to transform two units of good
1 into four units of good 2, for that it must also have $\xi'_i = (-2,
4)$ in $\Xi_i$. It might be the case, for example, that companies get more efficient
with production and therefore it might have $\xi''_i = (-2, 6)$ in its
production set.

In General Equilibrium Theory, an economy is formally defined as the tuple

\begin{equation}
  E = \left(\{(X_j,U_j)\}_{j=1}^J,
    \{\Xi_i\}_{i=1}^N, \{\omega_j\}_{j=1}^J \right).\label{eq:econGE}
\end{equation}

One of the theory's assumptions is that the economy described is
\textbf{complete}, that is, every agent can exchange every good with no
transaction costs and complete information about the firm's
technologies, other consumer's consumption, etc. Also, a good $\mu$
contains all the possible information that a consumer would take into
account when making his choice. That is, among the space of goods we
could have ``umbrella'' and ``chocolate'', or we could also have ``an
umbrella on August 13th, 2016 in Sao Paulo with 50\% chance of rain''
and ``an umbrella on December 12th, 2016 in Chicago with 90\% chance
of rain''. 

It's assumed that agents are \textbf{price-takers}, that is, they are
unable to affect the market prices and therefore take them as a
given. The prices of the goods are given by a $M-$dimensional vector
$p = (p_1, \ldots, p_M)$, where each price is a strictly positive
quantity, ie, $p_\mu>0$ for all $\mu$. This assumes that goods have global
prices, which is consistent with the completeness assumption: there is
no reason why the market prices should be different for certain
consumers or firms if they have complete knowledge and no transaction
costs.

With a price vector $p$ defined, we say the consumer $j$ has a budget
$B_j = p\cdot \omega_j$, which is the monetary value of his initial
endowment. Any bundle he chooses to purchase will cost him
$p\cdot x_j$. His objective, therefore, is to find the best bundle
$x_j$ he is able to afford, that is:

\begin{equation}
  \label{eq:consumer_obj}
  \max_{x_j \in X_j} U(x_j), \quad \text{s.t. } p\cdot x_j \leq p\cdot \omega_j
\end{equation}

The firms, on the other hand, have an operating profit for each
technology given by $p \cdot \xi_i$, which is how much money they earn
by selling their outputs ($\xi_i^\mu > 0$) minus how much they spend
purchasing the inputs ($\xi_i^\mu < 0$). Their objective is to maximize
their profits, that is:

\begin{equation}
  \label{eq:firm_obj}
  \max_{\xi_i \in \Xi_i} p \cdot \xi_i
\end{equation}

With these ingredients laid out, we define an \textbf{allocation} of the
economy as a set of specific choices for consumption bundles and
technologies, ie, an allocation $a$ of an economy $E$ is

\begin{equation}
  \label{eq:3}
  a = (x_1, \ldots, x_J, \xi_1, \ldots, \xi_N), \, x_j \in X_j, \,
  \xi_i \in \Xi_i
\end{equation}

The economy is closed, and therefore all that is produced must come
from the initial endowments and be consumed by the consumers. We
therefore say an allocation is \textbf{feasible} if it satisfies
\textbf{market clearing} for all the goods:

\begin{equation}
  \label{eq:market_clearing}
  \sum_{j=1}^J x^\mu_j = \sum_{j=1}^J \omega_j^\mu + \sum_{i=1}^N
  \xi_i^\mu, \quad \forall \, \mu \in \{1,\ldots, M\}
\end{equation}

This is a strong condition which constraints many quantities in
the economy. In particular, if we multiply both sides of the equation
by $p^\mu$ and sum then in $\mu$ we get, in vector notation,

\begin{equation}
  \label{eq:price_market_clearing}
  \sum_{j = 1}^J p \cdot (x_j - \omega_j) = \sum_{i=1}^N p\cdot \xi_i
\end{equation}

The left handside is the leftover money the consumers have after
making their choice of consumption, also called the value of excess
demand, whereas the right handside is the firms aggregate profit, also
known as the value of excess suply. Because we assume that the
consumer may not spend more than his budget, the value of each
consumer's individual excess demand has to be non
positive. Simultaneously, if we assume that the firms always have
$\xi_i = 0$ in their production set, ie, we assume that they can
always opt to not produce at all and leave the market, then the value
of excess supply for each firm has to be non negative. Because they
must be equal, we conclude that in an economy for which market
clearing holds, the consumer spends all his available budget and the
firms all have zero profit, a result known as \textbf{Walras' Law}.

Given a set of possible feasible allocations $\{a_k\}$, we may wonder
if there is any allocation we desire most over the other. This
of course depends on the criteria we use to judge them: we may like
allocations with less inequality, with the most aggregate utility,
with the smallest minimum utility, etc. Economist opt to use one
particular condition which is called \textbf{Pareto optmality}.

Intuitively, a \textbf{Pareto optimal} (or \textbf{Pareto efficient}
allocation is one that you can't make a consumer better without making
another consumer worse off. The idea is that, a non Pareto optimal
allocation has some waste in it: one could change the consumption
bundles in order to increase some utilities and no other consumer
would complain. Because firms have zero profit in feasible
allocations, they wouldn't mind the change. 

More formally, a feasible allocation $a = (x, \xi)$ is said to be
\textbf{Pareto optimal} if there is no other allocation that
\textbf{Pareto dominates} it, that is, no allocation $a' = (x', \xi')$
such that $U(x_j') \geq U(x_j)$ for all $j$ and $U(x_j') > U(x_j)$ for
at least one $j$.

The Pareto optimality concept therefore defines a socially desireble
outcome in a ``non-controversial'' way, by definition no agent in the
economy would have a problem with policies or actions taken to make it
more Pareto efficient. However, it says nothing about equality: an
allocation in which one consumer has all the goods and no other
consumer has any goods is Pareto optimal.

We finally arrive to the concept of equilibrium in an economy. A
\textbf{Walrasian equilibrium} (or competitive equilibrium or simply
equilibrium) in an economy E is an allocation $(x^\ast, \xi^\ast)$ and a
price vector $p$ such that

\begin{enumerate}
\item Every firm $i$ maximizes it's profits in its production set
  $\Xi_i$, that is
  \begin{equation}
    \label{eq:4}
    p\cdot \xi_i^\ast \geq p\cdot \xi_i, \quad \forall \xi_i \in
    \Xi_i,\quad \forall i in \{1, \ldots, N\} 
  \end{equation}
\item Every consumer $j$ maximizes his utility in his consumption set
  $X_j$, that is
  \begin{equation}
    \label{eq:1}
    U(x_j^\ast) \geq U(x_j), \quad \forall x_j \in X_j, \quad \forall
    j \in \{1,\ldots, J\}
  \end{equation}
\item The allocation $(x^\ast, \xi^\ast)$ is feasible, that is,
  \begin{equation}
    \label{eq:2}
    \sum_{i=j}^J x_j^\ast = \sum_{j=1}^J \omega_j + \sum_{i=1}^N \xi_i^\ast
  \end{equation}
\end{enumerate}

The Walrasian equilibrium is essentially a pair allocation - prices in
each all the optimization problems are solved at once. Although we
have made no mention of dynamics in this economy, it's considered an
equilibrium because all agents are as satisfied as possible with their
allocation given the prices, which we have assumed to be global and
unchangeable by any agent's action. This is not exactly a definition
of equilibrium as used in Physics, but we will discuss this point
later. For now, we point out that a Walrasian equilibrium is in some
sense stable.

One 

We have thus defined two desirable properties of an allocation:
efficiency and equilibrium. The fundamental results of General
Equilibrium Theory are the \textbf{welfare theorems}, which define the
conditions in which an equilibrium is Pareto optimal and vice versa.

The \textbf{First Fundamental Welfare Theorem} asserts that if the consumers have
a utility function continuous on $X_j$\footnote{The actual theorem
  asserts a weaker condition, that the preferences be locally
  nonsatiated, that is, for every $x \in X$, there is an $x' \in X$
  such that $\|x - x'\| < \varepsilon$ and $x'$ is preferred to $x$.},
then all Walrasian equilibria are Pareto optimal. This result is
simple yet useful, because it tells us that if our economy is in
equilibrium, we don't have to care about checking if it's
efficient. The violation is also important: if a given economy we are
studying is in an inefficient equilibrium, then it must be that one of
the theorem's condition was violated. This sheds light in where to
look for market failures. We remind the reader, however, that some
extra strong assumptions were made for the economies described by this
therorem, namely, completeness of market and global prices that
no single agent is capable of influencing.

The \textbf{Second Fundamental Welfare Theorem} requires extra
assumptions: it afirms that if an economy satisfies the condition of
the first fundamental theorem, the utility functions $U_j$ plus all
sets $X_j$ and $Y_i$ are convex and if we are able to redistribute the
initial endowments at will while keeping the total amount
$\sum_{j=1}^J \omega_j$ constant, then for every Pareto efficient
allocation there exists a wealth allocation $\omega$ and price vector
$p^\ast$ such that $(x^\ast, \xi^\ast, p^\ast)$ is a Walrasian
equilibrium.

The second theorem is considerably more interesting than the first
one: any Pareto optimal allocation we would like in an economy can be
an equilibrium given the appropriate price vector and a possible
wealth transfer, albeit under a stronger set of conditions.

\section{The Dynamics of General Equilibrium}

A conspicuous element was missing from the exposition above: there are
no rules for the dynamics of the economies described above. The prices
are taken as a given, as are the consumer and firm choices. What
happens if a firm closes? What happens if a new firm appears? The
equilibrium simply ``recalculates'' and the economy moves to the new one?

Indeed, this is a long standing criticism of General Equilibrium
Theory. Walras proposed it as a process of
\textbf{tatônnement}\footnote{From ``trial and error'' in french}: a
central figure, known as the Walrasian auctioneer, sugests a price and
asks all the firms and consumers how much would they like to produce
and buy at these given prices, but without any transaction taking
place at the out of equilibrium prices. The auctioneer updates the
prices in the direction of diminishing excess demand or supply, a
``gradient descent'' of sorts, until equilibrium is reached.

However, this process is quite indetermined. Chiefly, this auctioneer
figure doesn't exist in most decentralized markets: goods are traded
at agreed prices by both parts, which do not wait until their
transaction is authorized by some central authority. Even if they did,
such an auctioneer would require an infinitely large computational
capability to compute the excess demand and supply of every consumer
and firm and for every good in a modern economy \cite{axtell05,
  Roughgarden10}. Worst of all, even if there was such central figure
with such an arbitrary large amount of computing power, the price
updating dynamic is not guaranteed to
converge \cite{Gintis07}. Finally, even if it converges, we have no
assurance that it will converge in finite time.




%computable general equilibrium

% general equilibrium theory
% def of walrasian eq, market clearing, welfare theorems
% criticisms
% diff between economic equilibria and physics equilibria
\cite{thurner} \cite{Schumpeter39} \cite{savage} \cite{mascolell}

\cite{Tversky74, Kahneman79, Tversky81}
