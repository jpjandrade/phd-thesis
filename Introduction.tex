\begin{quote}
The wide perspective opening up, if we think of applying this science
to the statistics of living beings, human society, sociology and so
on, instead of only to mechanical bodies, can here only be hinted at
in a few words. \cite{Boltzmann}
\end{quote}


Boltzmann's expectation that statistical mechanics has potential applications far beyond those in physics relies on the fact that distributions of aggregate (e.g. sums) variables converge to limit distributions (e.g. the Gaussian, in the case of the central limit theorem).  The textbook example under which these results hold -- that of independent and identically distributed variables -- is prominently inappropriate in economics, because interactions are not necessarily weak and because of the persistent heterogeneity that pervades economic activity. For example, size distributions (e.g. firms or of individuals in terms of wealth or income) are remarkably skewed \cite{firm-sizes, firm-sizes2, wealth-inequality, wealth-inequality2}, suggesting that macro-phenomena might be related to the behaviour of few ``big'' players rather than arise as collective properties of a population.

\section{Main Results}